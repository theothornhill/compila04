\usepackage[utf8]{inputenc}
\usepackage[T1]{fontenc}

\usepackage[normalem]{ulem}  %%% redefines emph!!!!!!!!!!!!!!!!!!!!!!
\usepackage{soul}
\usepackage{import}                      %% relative path in "inputs"
\usepackage{etex}
\usepackage{graphics}
\graphicspath{{./figures/}}                %% cool!
\usepackage{psfrag}
\usepackage{cancel}
\usepackage{MnSymbol}    %% for liveness 

\usepackage{xcolor}
\definecolor{electricblue}{HTML}{05ADF3}


\usepackage{amsmath}

\usepackage[dvipsnames]{colortbl}
\def\grayscale{0.8}                              %% defaul gray
\newcommand{\gcell}{\cellcolor[gray]{\grayscale}}   %% gray cell                   

\def\scalefactor{0.6}                     %% for pictures
\usepackage{tikz}
\usepackage{tikz-qtree}                   %% for more elegant trees
\usetikzlibrary{matrix,arrows}  %%% for diagrams
\usetikzlibrary{chains,fit,shapes}
\usetikzlibrary{arrows,automata}
\usetikzlibrary{graphs}
%\usepackage[nocolor]{drawstack}   %% currently does not work 
%\usepackage{drawstack}


\usepackage{cancel}

\usetikzlibrary{arrows, decorations.markings}


%%%%%%%%%%%%%%%%% for LR-NFA's 
\tikzstyle{inititem}=[fill=blue!10]
\tikzstyle{completeitem}=[fill=red!10]
\tikzstyle{conflictitem}=[fill=red!45]
\tikzstyle{initcompleteitem}=[fill=violet!20]
%%%%%%%%%%%%%%%% for stacks

\tikzstyle{sframerest}=[fill=black!05]
\tikzstyle{sframe0}=[fill=orange!10]
\tikzstyle{sframe1}=[fill=blue!10]
\tikzstyle{sframe1p}=[fill=blue!20]
\tikzstyle{sframe2}=[fill=green!10]
\tikzstyle{sframe2p}=[fill=green!20]
\tikzstyle{sframe3}=[fill=red!10]

\usepackage{listings}[2003/06/21] % for typesetting code, take care of the version

\iffalse
\lstloadaspects{keywordcomments}  %% hack for algol
\usepackage{lstassem}
\usepackage{lsttiny}
\usepackage{lsttac}
\usepackage{lstpcode}
\usepackage{lstasurmcode}   %% old dragon book register machine code
\usepackage{lstrmcode}      %% new dragon book register machine code
%\lstset{language=tac}
\usepackage{lstpseudo}
\lstset{mathescape=true}
\lstset{language=pseudo} 
%\lstset{basicstyle=\tt}
\lstset{basicstyle=\scriptsize}
\lstset{frame=trBL}

\lstset{numberstyle=\tiny,numberfirstline=true,stepnumber=1,numbers=left}
\lstset{emphstyle={\bfseries\textcolor{beamerblue}}}
\lstset{emphstyle=[2]{\bfseries\textcolor{beamerred}}}
\fi

\usepackage{makeidx}
\makeindex






\usepackage{varioref}
\newtheorem{remark}            {Remark} %% only in slides


\usepackage{makecell}

\ifslides  %% I set the minitoc depth
\newcounter{minitocdepth}
\fi


\iffalse  %%%%% rest %%%%%%%%%%%%%%%%%%%%%%%%%%%%%%%%%%%%%%%%%%%%%%%%%%%%%%%%%%%



%XXX\usepackage{multicol}





%\usepackage{xcolor}
%\PassOptionsToPackage{hyperref,x11names}{xcolor}
%% electric blue = #2C75FF?

%\usepackage{tocloft}
%\renewcommand{\cftsecleader}{\cftdotfill{\cftdotsep}}
%\usepackage[breaklinks=true,xetex]{hyperref} 
% \usepackage{hyperref}   %% Does not work in the exam
%\hypersetup{colorlinks}
%\hypersetup{colorlinks, citecolor=electricblue,filecolor=electricblue,linkcolor=electricblue,urlcolor=electricblue}



%\usepackage{bm}





\ifhandout\else


%%%%%%%%%%%%%%%%%%%%%%%%%%%%%%%%%%% Beamer styling %%%%%%%%%%%%%%%%%%%%%%%%%%%%%%%%%%%%%%%%%%%%%
\setbeamercolor{alerted text}{fg=beamerred}
\useinnertheme[shadow]{rounded}
\setbeamertemplate{sections/subsections in toc}[sections numbered]  %% I don't like the bullets

\definecolor{Descitem}{RGB}{0, 0, 139}

\definecolor{StdTitle}{RGB}{26, 33, 141}
\definecolor{StdBody}{RGB}{213,24,0}

\definecolor{AlTitle}{RGB}{255, 190, 190}
\definecolor{AlBody}{RGB}{213,24,0}

\definecolor{ExTitle}{RGB}{201, 217, 217}
\definecolor{ExBody}{RGB}{213,24,0}

\setbeamercolor{block title}{fg = Descitem, bg = StdTitle!15!white}
\setbeamercolor{block title}{fg=red!67!black,bg=blue!06!white}

\AtBeginSection[]%
{%
\begin{frame}[plain]%
  \begin{center}%
    \usebeamerfont{section title}\insertsection%
  \end{center}%
\end{frame}%
}





\useoutertheme{split}
\useoutertheme{shadow}

\setbeamercolor{frametitle}{fg=red!67!black,bg=blue!06!white}



% \setbeamercolor{alerted text}{fg=beamerred}
% \useinnertheme[shadow]{rounded}
% \setbeamertemplate{sections/subsections in toc}[sections numbered]  %% don't like the bullets

% \definecolor{Descitem}{RGB}{0, 0, 139}

% \definecolor{StdTitle}{RGB}{26, 33, 141}
% \definecolor{StdBody}{RGB}{213,24,0}

% \definecolor{AlTitle}{RGB}{255, 190, 190}
% \definecolor{AlBody}{RGB}{213,24,0}

% \definecolor{ExTitle}{RGB}{201, 217, 217}
% \definecolor{ExBody}{RGB}{213,24,0}

% \setbeamercolor{block title}{fg = Descitem, bg = StdTitle!15!white}
% \setbeamercolor{block title}{fg=yellow!67!black,bg=blue!02!white}
% \setbeamercolor{frametitle} {fg=yellow!67!black,bg=blue!02!white}

%\usetikzlibrary{automata}
%\usetikzlibrary{shadows}      %% drop shadow
%\usetikzlibrary{positioning}


\fi


%\newcommand{\bcell}{}


\usepackage{comment}
\ifhandout\else
\pgfdeclareimage[height=2cm,interpolate=true]{uio}{../tex/inputs/uiologo}%%
\titlegraphic{
  \begin{center}
    \pgfuseimage{uio}
  \end{center}}
\fi


\ifhandout
\usepackage[draft]{fixme}[2013/01/28]

%\fxusetheme{color}






\renewcommand\fxenglishnotename{}
%\fxsetup{mode=}
%\fxsetup{targetlayout=color}
\fxsetface{margin}{\tiny\textcolor{red}}
%\fxsetface{env}{\XXXolor}
\fxsetface{inline}{\color}
\fxsetface{target}{\color}
%\fxsetup{envface=\scriptsize\tt\textcolor{red}}  %% must be after the color
%\fxusetheme{color}

\FXRegisterAuthor{ms}{ems}{ms}
\else

\newenvironment{emserror}{}{}
\newenvironment{emsnote}{}{}



\usepackage{textcomp}
%\excludecomment{emsnote}
%\excludecomment{emserror}
\fi

\setcounter{secnumdepth}{3}
\setcounter{tocdepth}{2}
%\usepackage{multirow}  %%  vertical multicolumn


% \AtBeginSection[]%
% {%
% \begin{frame}[plain]%
%   \begin{center}%
%     \usebeamerfont{section title}\insertsection%
%   \end{center}%
% \end{frame}%
% }


% \AtBeginSection[]{%
%   \begin{frame}<beamer>
%     \frametitle{Outline}
%     \tableofcontents[sectionstyle=show/hide,subsectionstyle=hide/show/hide]
%   \end{frame}
%   \addtocounter{framenumber}{-1}% If you don't want them to affect the slide number
% }


\usepackage{url}

%\usepackage{epsfig}
\input{prooftree}
%\usepackage{bussproofs}

\ifhandout\else
\beamertemplatedotitem
\beamertemplatetextbibitems     %% real citations!
\beamertemplatetransparentcoveredmedium
\setbeamertemplate{blocks}[rounded][shadow=true]


\setbeamercolor{alerted text}{fg=beamerred}
\useinnertheme[shadow]{rounded}

%\AtBeginSection[]%
%{%
%\begin{frame}[plain]%
%  \begin{center}%
%    \usebeamerfont{section title}\insertsection%
%  \end{center}%
%\end{frame}%
%}


%\AtBeginSubsection[]
%{ 
%  \begin{frame}<beamer>
%    \frametitle{Outline}
%    \tableofcontents[currentsection,currentsubsection]
%  \end{frame}
%}

%\X
\setbeamertemplate{section page}
{
    \begin{centering}
    \begin{beamercolorbox}[sep=12pt,center]{part title}
      \Large \inserttitle 
      \\[2em]
      \normalsize \insertsection 
      \\[1em]
      \normalsize \insertdate 
    \end{beamercolorbox}
    \end{centering}

  \begin{center}
    \pgfuseimage{uio}
  \end{center}
}


\AtBeginSection[]{%
  \frame{\sectionpage}
\begin{frame}<beamer>
    \frametitle{Outline}
    \tableofcontents[currentsection,currentsubsection]
  \end{frame}
  
}%

\AtBeginSubsection[]{%
\begin{frame}<beamer>
    \frametitle{Outline}
    \tableofcontents[currentsection,currentsubsection]
  \end{frame}
  
}%


\useoutertheme{split}
\useoutertheme{shadow}




\setlength{\itemsep}{9mm}
%\setbeamertemplate{itemize item}{$\Rightarrow$}



\setbeamertemplate{footline}[page number] %%% empty footline
\setbeamertemplate{headline}[default] %%% empty headline
\setbeamertemplate{enumerate items}[default]



\fi




%\AtBeginSection[]{\tableofcontents[current]\mbox{}}
%\usetheme{Malmoe}  %%% MALMOE does not work well with
                    %%  the tree overlays.


%\usepackage{mathptm}
%\usepackage{utopia} % utopia does not work  pifont euler bookman avant
%\useoutertheme{infolines}

%%%%





%\ifhandout
%
%\newtheorem{remark}            {Remark} %% only in slides
%\else
%
%\excludecomment{remark}
%\fi

\newcommand{\finish}[1]{}
\usepackage{graphicx}
\graphicspath{{./figurer/}}    %% cool!

\usepackage{textcomp}


%\usepackage{pgf,pgfpages}

%%\pgfpagesuselayout{4 on 1}[a4paper,landscape,border shrink=5mm] \nofiles
%\usepackage{xspace}

%\usepackage{alltt}




%\usepackage[upright]{fourier}
%\usepackage{tkz-graph}
%\usepackage{tkz-berge}



%\usepackage{fourier}
%\usepackage{tikz}
%\usetikzlibrary{arrows,%
%               shapes,positioning}


%\usetikzlibrary{shadows}
%usetikzlibrary{shapes.geometric}

% \usetikzlibrary{shapes.multipart}
% \usetikzlibrary{arrows,automata}


%%%% It's a hack for parially uncovering graphs (trees etc)
%%%% opacity = 0 means: invisible
%%%% be careful, it might not work, some pdf-latex
%%%% seem to have problems with it.
% \tikzset{
%   invisible/.style={opacity=0},   
%   visible on/.style={alt=#1{}{invisible}},
%   alt/.code args={<#1>#2#3}{%
%     \alt<#1>{\pgfkeysalso{#2}}{\pgfkeysalso{#3}} % \pgfkeysalso doesn't change the path
%   },
% }

%tikzset{hide on/.code={\only<#1>{\color{white}}}} %% just the node content


%XXXX\usepackage{color}


%\lstset{basicstyle=\color{rawsienna}\footnotesize,%
%        frame=single,mathescape, inputencoding=latin1, extendedchars}
%\lstset{emphstyle=\bf\underline}%



%\lstset{basicstyle=\color{rawsienna}, fontsize=\footnotesize,%
%        frame=single,mathescape, inputencoding=latin1, extendedchars}


%\lstset{emphstyle=\bf\underline}






\makeatletter
\newcommand*{\overlaynumber}{\number\beamer@slideinframe}
\makeatother



\makeatletter
%newenvironment<>{btHighlight}[1][]
%\begin{onlyenv}#2\begingroup\tikzset{bt@Highlight@par/.style={#1}}\begin{lrbox}{\@tempboxa}}
%\end{lrbox}\bt@HL@box[bt@Highlight@par]{\@tempboxa}\endgroup\end{onlyenv}}

%newcommand<>\btHL[1][]{%
% \only#2{\begin{btHighlight}[#1]\bgroup\aftergroup\bt@HL@endenv}%
%
%def\bt@HL@endenv{%
% \end{btHighlight}%   
% \egroup
%
%newcommand{\bt@HL@box}[2][]{%
% \tikz[#1]{%
%   \pgfpathrectangle{\pgfpoint{1pt}{0pt}}{\pgfpoint{\wd #2}{\ht #2}}%
%   \pgfusepath{use as bounding box}%
%   \node[anchor=base west, fill=yellow!10,outer sep=0pt,inner xsep=1pt, inner ysep=0pt, rounded corners=3pt, minimum height=\ht\strutbox+1pt,#1]{\raisebox{1pt}{\strut}\strut\usebox{#2}};
% }%
%
%makeatother



%\lstset{moredelim={**[is][\btHL<1>]{@1}{@}},
%  moredelim={**[is][{\btHL<2>}]{@2}{@}},name=X,
%  basicstyle=\scriptsize}






%%{[2]\color{rawsienna}\footnotesize}


\newcommand{\npl}{\mathit{npl}}
%\fi




%%% The chain library is for doing ``chains'' of nodes.

\usepackage{calc}
\usetikzlibrary{calc,trees,positioning,arrows,chains,shapes.geometric,%
    decorations.pathreplacing,decorations.pathmorphing,shapes,%
    matrix,shapes.symbols}

%%%% The following is the style for the compiler architecture figure.
%%% I found stuff on the net and adapted it.


\tikzstyle{every node}=[font=\scriptsize] 
\tikzset{
>=stealth',
  chainelement/.style={
    rectangle, 
    rounded corners, 
    fill=black!10,
    draw=black, very thick,
%    font=\footnotesize,
    text width=10em, 
    minimum height=1.5em, 
    text centered, 
    on chain},
  line/.style={draw, thick, <-},
  element/.style={
    tape,
    top color=white,
    bottom color=blue!50!black!60!,
    minimum width=8em,
    draw=blue!40!black!90, very thick,
    text width=10em, 
    minimum height=3.5em, 
    text centered, 
    on chain},
  every join/.style={->, thick,shorten >=1pt},
  decoration={brace},
  tuborg/.style={decorate},
  tubnode/.style={midway, left=2pt},  %%left=position of the node (= text)
}

\fi %%%%

%\usepackage[matrix,arrow,curve,all,2cell,ps]{xy}

%%% Local Variables: 
%%% mode: latex
%%% TeX-master: "main-script"
%%% End: 

