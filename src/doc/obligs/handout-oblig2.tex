\documentclass[11pt,freeform]{handout}[2014/08/13]

%\dozent{\mbox{}}
\def\myauthor{\href{http://www.ifi.uio.no/~msteffen}{Martin Steffen}}
\def\mytitle{INF 5110: Compiler construction}

\coursename{INF 5110: Compiler construction}
\semester{Spring 2019}
\dozent{Martin Steffen}
\ifigroup{Reliable system}


%%% Local Variables: 
%%% mode: latex
%%% TeX-master: t
%%% End: 

\newcommand{\deadlineone}   {Friday, 15. 03. 2019}
\newcommand{\deadlinetwo}   {--11. May  2018--}


%%% Local Variables: 
%%% mode: latex
%%% TeX-master: "handout-oblig1.tex"
%%% End: 



\pgfdeclareimage[height=2.4cm,interpolate=true]{uio}{uiologo}%% relative
%\usepackage[german]{babel} 
%\usepackage{german}

\handouttitle{Oblig 2}
\handoutnumber{2}
\issuedate{18.\ 04.\ 2018}
\topic{Oblig 2}




\begin{document}
\thispagestyle{empty}



\section{Official }
\label{sec:official-info}




\hrulefill{}

The \textbf{deadline/frist}  for the second oblig is

\begin{quote}
  \textbf{\deadlinetwo, 23:59}
\end{quote}





\section{What and how to hand in}
\label{sec:what-how}


\subsection{Git}
\label{sec:git}

You will continue in the group/ git-repos you did in the first oblig
(unless really convincing reasons speak against it).  Basically, you
continue with your previous code, add the new functionality, push a
solution before the deadline, and inform me when it's done so that I can
update. It's important to tell me, as I don't want to repeetedly update in
the hope that it's done.


If a change in arrangement is needed (merge of groups, or a split of
groups), you need to ask for that re-arrangement (not just that on the day
of the deadline it's ``announced'' that there is now a new group \ldots).


\subsection{What to include into a solution}
\label{sec:what-include-into}


\begin{itemize}
\item A top-level \emph{Readme-file} containing
  \begin{itemize}
  \item  containing names and emails of the authors
  \item instructions how to build the compiler and how to run it.
  \item test-output for running the compiler on \texttt{compila.cmp} as
    input
  \item of course, all code needed to run your package. That includes
    \begin{itemize}
    \item \textsl{JFlex}-code for the scanner
    \item \textsl{CUP}-cpde for the 2 variants of the syntax
    \item the Java-classes for the syntax-tree
    \item the build-script \texttt{build.xml}
    \end{itemize}
  \end{itemize}


\end{itemize}



\section{Purpose and goal}
\label{sec:x}

The goal of the task is to collect more practical experience implementing a
compiler, in particular, a taste of phases after parsing. It's only a
taste, as we don't have the time to get a full-scale compiler on its
feet. The language we are compiling is (as before) described in the
\emph{compila18 language specification.} This time, also the later sections
about type checking etc, what were irrelevant for oblig 1, specify the
scope of the task as far as the language features are concerned.


As in oblig 1, it's necessariy that a solution is equipped

\begin{quote}
  with ``automatic test-cases''
\end{quote}
That can be done (as before) via ant targets. Those tests have to be
executable on the RHEL linux pool at the university.\footnote{That should
  actually not be a big restriction, as Java (and the task) is to a big
  extent platform independent (``write once, run everywhere''
  \ldots). Nonetheless: Based on experience with the first obligs: it's
  advised to make this ``test'' setup early on (not after the deadline), to
  design the code \emph{with the goal that it runs also at a different
    place than one's own platform} and to test that this goal is actually
  met. The reason for that ``testability'' requirment is that correction
  will again not be based on reading code from my side, but in first
  approximation: running the test. In that sense, it's also not of primary
  importance, whether it's \texttt{ant} or perhaps \emph{make} or some
  script. Important is, that I can execute it at the RHEL machines by
  invoking a simple command.  I don't have the time to figure out how one
  particular solution is configured, started, etc. I don't even want to
  look around and try whether I find a \texttt{main} method somewhere\ldots
}






\section{Tools}
%\label{sec:tools}


The tools are basically the same as for the previous oblig, and typically
you will continue anyway with the previous set-up..




\section{Task more specifically: Type checking and }
\label{sec:task-more-spec}


The overal task is to 

\begin{quote}
  implement a parser for the \textsl{Compila 18} language.
\end{quote}
The language specification is given in a separate document. Oblig 1 is
concerned with checking \emph{syntactic correctness}, which means, not all
of the language specification is relevant right now: semantic correctness,
type checking etc. will become relevant only later for the second oblig.


\subsection{Syntax tree}
\label{sec:syntax-tree}

The result of a successful parse is an \emph{abstract syntax tree}. That
data structure needs to be appropriately ``designed''. In a Java
implementation, that involves the definition of appropriately chosen
classes arranged in some class hierarchy. Make also use if \emph{abstract
  classes}. In the lecture, there had been some ``design guidelines'' that
may be helpful. Carefully chosen names for classes will help in a
conceptually clear implementation. A definitely \emph{non-recommended} way
is to have one single class \texttt{Node} lumping together all kinds of
nodes and syntactic categories in the syntax tree.







\subsection{Print out of the AST}
\label{sec:print-out-ast}


The AST should be ``printed''. The easiest and recommeded form of printout
is in \emph{prefix form}. In the provided ``starting point'', there is an
example compila input file and a corresping file containing a possible
output. The two files are called

\begin{itemize}
\item \texttt{compila.cmp}
\item \texttt{compila.ast}
\end{itemize}


Note: the two files are meant as inspiration. Each year the syntax of
\textsl{compila} slightly changes (wrt. keywords, associativity etc). So it
might not 100\% in accordance with the 2018 version.

It's allowed (but not necessary) to print it in other forms than prefix
form. But the output must indicate the AST in readable form (``readable''
as in human-readble that is \ldots). Note, the task is not that the output
is a syntactically correct \textsl{compila} program again (that might be a
formatting tool), we just need a way to look at the syntax tree, which
comes in handy for debugging,


%\bibliographystyle{apalike}
%{\small
% \bibliography{string,semantics,crossref}
%% \bibliography{extracted}
% }

\subsection{Two grammars}
\label{sec:two-grammars}


As mentioned shortly, the task requires 2 grammars, representing 2 ways
dealing with precendence and associativity.

\begin{enumerate}
\item an \emph{unambiguuous} grammar resolving precedence and associativity
  by ``baking it in'' directly into the grammar. The grammar is in plain
  BNF (in the form required by the tools)
\item the second grammar is ambiguous and relies on \emph{CUP} to resolve
  the associativity and precendence. This second grammar will probably look
  nicer and will be shorter. It's therefore probably best to take that one
  as \emph{default} (for instance for oblig 2).
\end{enumerate}

\subsubsection*{Comparison and discussion}
\label{sec:comparison}

Investigate and characterise \emph{conflicts} of the \emph{original}
grammar. How many states do the 2 generated CUP grammars have? That
requires a look into the CUP-generated code. Discuss also whether the
choice of the two grammars influences the generation of the AST: is one of
the two approaches easier to work with when it comes to generate an AST
(resp. your chosen AST data structure.


\textbf{Note:} It's not required to provide code to build \emph{two}
versions of AST-generation, one is enough. In other words, for one of the
two grammars, you don't need ``action code'' in the grammar to produce an
AST, plain \emph{checking} is sufficient.

\subsection{Lexical analysis}
\label{sec:lexical-analysis}

As mentioned, \textsl{JFLex} is the tool of choice for lexical analysis. It
delivers a token to the parser via the method \texttt{next\_token()}.  

As far the the ``theoretical'' task concerning \textsl{compila 18} is
concerned, the lexer is responsible for ignoring comment, white-space etc,
find keywords and the like. 

Besides that, one has to make the parser and the lexer ``work together''
hand in hand. Information about that can be found in the corresponding
manual. There should also be examples for inspiration. A crucial ingredient
is the interface \texttt{java\_cup.runtime.Scanner} which needs to be
implemented by the actual scanner. The scanner will hand over tokens of the
type \texttt{Symbol} and one can use \texttt{Symbol.value} to pass ``text''
or other objects from the lexer to the parser.



\subsection{Error handling}
\label{sec:error-handling}

Error handling can be done simple: When hitting an error, parsing should
stop (as opposed to try to continue and give back an avalanche of
subsequent errors). Some meaningfull error message (at least wrt. which
syntactic class caused the error) would be welcome, as opposed to a plain
``sorry, bad program''. It's not required to give back line numbers
referrring to the original source code or positions in the original
file. In practice that's definitely useful (and not very hard either), but
not required for the oblig.




\section{Resources}
\label{sec:resources}

The course web-page contais an html file \texttt{resources.html} which
collects clickable links to \textsl{JFlex} \textsl{CUP} and corresponding
manuals.







\end{document}

Model Checking Cache Coherence Protocols for Distributed File Systems

%%% Local Variables: 
%%% mode: latex
%%% TeX-master: t
%%% End: 

%%% Local Variables: 
%%% mode: latex
%%% TeX-master: t
%%% End: 
